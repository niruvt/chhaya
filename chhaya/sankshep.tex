\छायांग{पुं}{पुल्लिंग}% Masculine
\छायांग{स्त्री}{स्त्रीलिंग}% Feminine
\छायांग{नपुं}{नपुंसकलिंग}% Neuter
\छायांग{१}{प्रथम व्यक्ती}% First person
\छायांग{२}{द्वितीय व्यक्ती}% Second person
\छायांग{३}{तृतीय व्यक्ती}% Third person
\छायांग{एव}{एकवचन}% Singular
\छायांग{द्विव}{द्विवचन}% Dual
\छायांग{त्रिव}{त्रिव}% Trial 
\छायांग{अव}{अल्पवचन}% Paucal
\छायांग{बव}{बहुवचन}% Plural
\छायांग{अवि}{अभिधानपर विभक्ती}% Nominative
\छायांग{कर्मवि}{कर्मपर विभक्ती}% Accusative
\छायांग{सा}{साधनपर विभक्ती}% Instrumental
\छायांग{दावि}{दानपर विभक्ती}% Dative
\छायांग{वियो}{वियोगपर विभक्ती}% Ablative
\छायांग{संयो}{संबंधयोजक विभक्ती}% Genitive
\छायांग{अधि}{अधिकरण विभक्ती}% Locative
\छायांग{संबो}{संबोधन विभक्ती}% Vocative
\छायांग{साह}{साहचर्यदर्शक विभक्ती}% Associative
\छायांग{कवि}{कर्तृत्वपर विभक्ती}% Ergative
\छायांग{आवि}{आगत विभक्ती}% Oblique
\छायांग{साक्रि}{साहाय्यक क्रियापद}% Auxiliary
\छायांग{गणक}{गणक}% Counter
\छायांग{भूत}{भूतकाळ}% Past
\छायांग{वर्त}{वर्तमान काळ}% Present
\छायांग{भवि}{भविष्यकाळ}% Future
\छायांग{पू}{पूर्ण}% Perfective
\छायांग{अपू}{अपूर्ण}% Imperfective
\छायांग{नि}{नित्य}% Habitual
\छायांग{अखं}{अखंडित}% Continuous
\छायांग{क्र}{क्रमिक}% Progressive
\छायांग{अक्र}{अक्रमिक}% Non-progressive
% 
% मराठी व्याकरणात वापरण्यात येणाऱ्या संज्ञांची छायांगसूची. ह्यात वरच्या यादीत उपस्थित असलेली
% छायांगे टाळली आहेत. मराठी व्याकरणातील पुढील संक्षेपांची यादी २०२१/०६/१३ रोजी आज्ञासंचाच्या
% आवृत्तिक्रमांक ०.३मध्ये सुशान्त देवळेकर ह्यांनी जोडली.
% 
% नामिक ह्या गटासंदर्भातील संज्ञा
\छायांग{नामि}{नामिक}
\छायांग{ना}{नाम}
\छायांग{प्राति}{प्रातिपदिक}
\छायांग{प्रत्य}{प्रत्यय}
\छायांग{सारू}{सामान्य रूप}% Oblique form
\छायांग{आब}{आदरार्थी बहुवचन}
\छायांग{प्र}{प्रथमा}
\छायांग{द्वि}{द्वितीया}
\छायांग{तृ}{तृतीया}
\छायांग{चतु}{चतुर्थी}
\छायांग{पं}{पंचमी}
\छायांग{ष}{षष्ठी}
\छायांग{सप्त}{सप्तमी}
% 
\छायांग{वि}{विशेषण}
\छायांग{गोवि}{गोड-गणातील विशेषण}
\छायांग{पांवि}{पांढर-गणातील विशेषण}
\छायांग{विवि}{विकारी विशेषण}
\छायांग{अविवि}{अविकारी विशेषण}
% 
% 
% धातु ह्या गटाशी संबंधित संज्ञा
\छायांग{धा}{धातु}
\छायांग{कृ}{कृदन्त}
\छायांग{धासा}{धातुसाधित}
\छायांग{क्रि}{क्रियापद}
\छायांग{कर्त}{कर्तरी}
\छायांग{कर्म}{कर्मणि}
\छायांग{भा}{भावे}
\छायांग{शक्य}{शक्यार्थक}
\छायांग{प्रयो}{प्रयोजक}
% 
% क्रियाविशेषणादी गटातील संज्ञा
\छायांग{क्रिवि}{क्रियाविशेषण}
\छायांग{के}{केवलप्रयोगी}
\छायांग{शयो}{शब्दयोगी}
\छायांग{उद्गा}{उद्गारवाचक}
\छायांग{अव्य}{अव्यय}